% Says jme:
% "The EFF tech report should go under a section called "Technical Reports."
% "you should list your talks under a section called "Presentations."

% resume.tex
% (C) 2003-2008 Asheesh Laroia; based on work by
% (c) 2002 Matthew M. Boedicker <mboedick@mboedick.org> http://mboedick.org
% $Id: resume.tex,v 1.6 2003/01/23 16:35:36 mboedick Exp $
%
\documentclass[10pt]{article}
\usepackage{fullpage}
\usepackage[left=2cm,top=1.2in,right=2cm,nohead,nofoot]{geometry} 
\pagestyle{empty}
\raggedbottom
\raggedright
\setlength{\tabcolsep}{0in}
\begin{document}
  \begin{tabular*}{6.5in}{l@{\extracolsep{\fill}}r}
    \textbf{Asheesh Laroia}  & asheesh@asheesh.org  \\
    & \\
    Address for correspondence: \\ 
    171 Second Street \#300 \\
    San Francisco, CA 94105  \\
    (585) 506-8865  \\
  \end{tabular*}
  \\
  \vspace{0.1in}
	 {\large \textbf{Education}}
	 \begin{itemize}
	 \item 
	   \begin{tabular*}{6in}{l@{\extracolsep{\fill}}r}
	     \textbf{Johns Hopkins University} & September 2002 - May 2007
	   \end{tabular*}
	   \begin{itemize}
	   \item May 2007: Master of Science in Engineering in Computer Science
	   \item May 2006: B.A. in Cognitive Science \\
	     Minors in Women, Gender, \& Sexuality; French Literature; and Computer Science \\
             Special Service Award from Computer Science (2006)
	   \end{itemize}
	 \end{itemize}

	 {\large \textbf{Work experience}}
	 \begin{itemize}

% CC ...           

	 \item 
	   \begin{tabular*}{6in}{l@{\extracolsep{\fill}}r}
	     \textbf{Software engineer} & July 2007 - ongoing  \\
	     Creative Commons & San Francisco, CA \\
	   \end{tabular*}
	   \begin{itemize}
	   \item Hired to maintain and build new Creative Commons tech projects, including a metadata library in C (liblicense) and a web API in JavaScript.
           \item Branched out to perform various technical tasks for Creative Commons, including security reviews of custom Creative Commons code and writing a custom Apache module.
           \item Regularly speak about Creative Commons: in Tokyo, about Science Commons; at the O'Reilly Open Source Conference in Portland, about liblicense; and generally about Creative Commons at numerous other venues.
           \item Mentored interns and Google Summer of Code students across two summers, leading to successful completion of various projects.
           \item Current primary focus: Firefox add-on for displaying Creative Commons-related metadata.
	   \end{itemize}
           
         \item
	   \begin{tabular*}{6in}{l@{\extracolsep{\fill}}r}
	     \textbf{Course assistant: Declarative Methods} & January 2007 - May 2007 \\ 
	     Johns Hopkins University & Baltimore, MD
	   \end{tabular*}
	   \begin{itemize}
           \item Graded assignments and held office hours tutoring students.
	   \end{itemize}


         \item
	   \begin{tabular*}{6in}{l@{\extracolsep{\fill}}r}
	     \textbf{Course assistant: Natural Language Processing} & September 2006 - December 2006 \\
	     Johns Hopkins University & Baltimore, MD
	   \end{tabular*}
	   \begin{itemize}
           \item Graded assignments and held office hours tutoring students.
	   \end{itemize}

	 \item 
	   \begin{tabular*}{6in}{l@{\extracolsep{\fill}}r}
	     \textbf{Technology intern and contractor} & June 2006 - July 2007  \\
	     Creative Commons & San Francisco, CA \\
	   \end{tabular*}
	   \begin{itemize}
	   \item Began as a technical intern in 2006, working primarily on a tool for feedback on the ``non-commercial'' clause of some Creative Commons licenses and gathering and analyzing statistics on the Creative Commons usage on the web.
           \item Stayed on part-time to continue developing these projects.
	   \end{itemize}

	 \item 
	   \begin{tabular*}{6in}{l@{\extracolsep{\fill}}r}
	     \textbf{Research assistant} & September 2004 - May 2007  \\
	     JHU Natural Language Processing laboratory & Baltimore, MD \\
	   \end{tabular*}
	   \begin{itemize}
	   \item Co-maintained NLP lab computing cluster. Customized, supported, and maintained MediaWiki and RT ticket tracker for use by NLP lab members, hardening both against spam and backing up their contents nightly.
	   \item Pursued an automatic PDF export process via \LaTeX{} for editing Dyna documentation book as wiki pages.
	   \end{itemize}
           
	 \item 
	   \begin{tabular*}{6in}{l@{\extracolsep{\fill}}r}
	     \textbf{Information/Communication Technology intern} & June  - August 2004  \\
	     World Food Programme Uganda office & Kampala, Uganda\\
	   \end{tabular*}
	   \begin{itemize}
	   \item Created database of supplementary feeding programmes using Microsoft Access and supported the use of the database in field offices.
	   \item Supported and maintained existing databases for school feeding programme.
	   \end{itemize}

%           \newpage

	 \item 
	   \begin{tabular*}{6in}{l@{\extracolsep{\fill}}r}
	     \textbf{Web Programmer} & September - December 2001  \\
	     Community Coalition for Long-Term Care & Rochester, NY\\
	   \end{tabular*}
	   \begin{itemize}
	   \item Developed basic informational web site for non-profit organization using PHP and standards-compliant HTML, CSS.
	   \end{itemize}
	 \end{itemize}

% Maybe put Eisner's name under JHU; same with Michalak
	 {\large \textbf{Research}}
	 \begin{itemize}
	 \item 
	   \begin{tabular*}{6in}{l@{\extracolsep{\fill}}r}
	     \textbf{Volunteer technical assistant} & February 2008 - April 2008\\
             Electronic Frontier Foundation & San Francisco, CA
	   \end{tabular*}
	   \begin{itemize}
% It would be nice to emphasize somehow that Seth brought me in because he was at a blocking point. Maybe I can just mention that in a personal statement.
% Eventually (AKA, ``when I write it,'' but hopefully really soon) I will write up a technical report to co-author with Seth.
	   \item Brought in by EFF staff technologist Seth David Schoen to solve problem of automating interactions with the US Patent Office Patent Application Information Retrieval (``PAIR'') system. After successfully extracting information on patent reexamination results beyond what is available in Lexis-Nexis, my work made contributed data to policy intern Raeanne Young's EFF June 2008 white paper, ``Patents and the Public Domain: Improving Patent Quality Upon Reexamination.''
	   \end{itemize}

% JHU real research

	 \item 
	   \begin{tabular*}{6in}{l@{\extracolsep{\fill}}r}
	     \textbf{Masters project} & October 2006 - May 2007\\
             Johns Hopkins University & Baltimore, MD
	   \end{tabular*}
	   \begin{itemize}
	   \item Produced under Dr. Jason Eisner and fellow student Eric Northup, as my masters project, an improved runtime library for the Dyna programming language, a declarative language for natural language processing algorithms.
	   \end{itemize}

% Earlier research projects that didn't pan out - should I ditch them from the resume?
% Probably the one with Rubin I should ditch, except that the one that appears
% on my transcript as ``Grade missing'' since we never finished.
% Sadly that same ``semester'' cognitive neuroscience totally crushed me, and I got a C-.
% (The ``semester'' is summer 2005.)

	 \item 
	   \begin{tabular*}{6in}{l@{\extracolsep{\fill}}r}
	     \textbf{Undergraduate research} & January 2006 - May 2006\\
             Johns Hopkins University & Baltimore, MD
	   \end{tabular*}
	   \begin{itemize}
	   \item Worked with Dr. Jason Eisner to process the full history of Wikipedia and create a corpus of style edits.
           \end{itemize}

% Nothing really came of this, sadly.
%	 \item 
%	   \begin{tabular*}{6in}{l@{\extracolsep{\fill}}r}
%	     \textbf{Johns Hopkins University} & June 2005 - August 2005\\
%	   \end{tabular*}
%	   \begin{itemize}
%	   \item Began work under Dr. Avi Rubin to analyze blog spam using machine learning algorithms. Work stalled when I was unable to get the cooperation of large blog services to donate data for a corpus.
%           \end{itemize}


	 \item 
	   \begin{tabular*}{6in}{l@{\extracolsep{\fill}}r}
	     \textbf{Summer undergraduate researcher} & July - August 2003\\
             University of Rochester & Rochester, NY
	   \end{tabular*}
	   \begin{itemize}
	   \item Worked with Dr. Lenhard K. Schubert and Phil Michalak on using the Google API and naive Bayes to perform word sense disambiguation as part of solving the knowledge acquisition bottleneck.
	   \end{itemize}
	 \end{itemize}
         
% Talks, whee

	 {\large \textbf{Presentations}}
	 \begin{itemize}
	 \item 
	   \begin{tabular*}{6in}{l@{\extracolsep{\fill}}r}
	     \textbf{Scrape the Web (tutorial)} & March 2009 \\
             PyCon 2009 & Chicago, IL
	   \end{tabular*}
	   \begin{itemize}
	   \item Invited to give a tutorial on ``programming websites that don't expect it''
           \item Discussed document tree navigation, web user-agent fingerprinting, CAPTCHAs, dealing with JavaScript, and ethics.
	   \end{itemize}

	 \item 
	   \begin{tabular*}{6in}{l@{\extracolsep{\fill}}r}
	     \textbf{Rights on the Desktop with liblicense} & July 2008 \\
             OSCON Open Source Convention & Portland, OR
	   \end{tabular*}
	   \begin{itemize}
	   \item Presented with Creative Commons CTO on the liblicense metadata library I maintain.
           \item Showed how it can be integrated into open source desktop applications with sample code for an image viewer, a media player, and a web photo gallery.
	   \end{itemize}

	 \item 
	   \begin{tabular*}{6in}{l@{\extracolsep{\fill}}r}
	     \textbf{Panel discussion on Creative Commons technology initiatives} & June 2008 \\
             Creative Commons Technology Summit & Mountain View, CA
	   \end{tabular*}
	   \begin{itemize}
	   \item As part of a discussion of Creative Commons metadata, presented to a diverse audience about the liblicense metadata library I maintain, explaining the need for machine-readable metadata and liblicense's approach.
	   \end{itemize}

	 \item 
	   \begin{tabular*}{6in}{l@{\extracolsep{\fill}}r}
	     \textbf{The Science Commons Neurocommons} & November 2007 \\
             National Academy of Sciences Workshop on
Designing Global Information & Tokyo, Japan \\
Commons for Innovation in Frontier Sciences
	   \end{tabular*}
	   \begin{itemize}
             \item Presented to a meeting of scientists in the Japan-U.S. Cooperative Science Program on the RDF-based ``open source knowledge management'' project, Neurocommons.
             \item Explained the core technical concepts behind RDF and how it can be used to bridge data sets from many sources. Demonstrated SPARQL queries that ask meaningful questions from our Neurocommons data and explained how data gets integrated.
	   \end{itemize}


	 \end{itemize}


         %	     \newpage
	 {\large \textbf{Technical activities}}
	 \begin{itemize}

% add Python teaching: EFF and outside

           
         \item
	   \begin{tabular*}{6in}{l@{\extracolsep{\fill}}r}
	     \textbf{Lead programming tutor} & September 2008 - ongoing \\
	     San Francisco Linux Users Group & San Francisco, CA
	   \end{tabular*}
	   \begin{itemize}
	   \item Currently lead a weekly introduction to programming (in Python) class for programming neophytes in the San Francisco Linux Users Group.
           \item Reviewed weekly assignments from students and improvised course material beyond the textbook.
	   \end{itemize}


         \item
	   \begin{tabular*}{6in}{l@{\extracolsep{\fill}}r}
	     \textbf{Python co-tutor} & March 2008 - ongoing \\
	     Electronic Frontier Foundation & San Francisco, CA
	   \end{tabular*}
	   \begin{itemize}
	   \item Co-teach weekly Python tutorial class with EFF staff technologist Seth David Schoen.
	   \end{itemize}

	 \item 
	   \begin{tabular*}{6in}{l@{\extracolsep{\fill}}r}
	     \textbf{Debian Maintainer} & September 2005 - ongoing \\
	     Debian GNU/Linux
	   \end{tabular*}
	   \begin{itemize}
	   \item Maintain packages used by thousands for premier Free Software distribution, earning Debian Maintainer status in December 2007.
           \item Collaborated with QA lead Lucas Nussbaum to improve an automated package quality assurance tool.
	   \end{itemize}

         \item
	   \begin{tabular*}{6in}{l@{\extracolsep{\fill}}r}
	     \textbf{Volunteer progammer} & January 2008  \\
	     Open Voting Consortium & California
	   \end{tabular*}
	   \begin{itemize}
	   \item Produced sample vote tabulation program for Open Voting Consortium demonstration to the San Luis Obispo County Democratic Party. The tabulator counted voter-verifiable paper receipts.
	   \end{itemize}



           

	 \item 
	   \begin{tabular*}{6in}{l@{\extracolsep{\fill}}r}
	     \textbf{Web team leader} & September 2005 - ongoing \\
	     Students for Free Culture
	   \end{tabular*}
	   \begin{itemize}
	   \item Handled many emergency repairs and sysadmin tasks involving database availability.
           \item Created and maintain centralized chapter web space service to bootstrap chapters' online presence.
	   \end{itemize}

	 \item 
	   \begin{tabular*}{6in}{l@{\extracolsep{\fill}}r}
	     \textbf{Core team member} & September 2004 - ongoing \\
	     Students for Free Culture
	   \end{tabular*}
	   \begin{itemize}
           \item After contributing written testimony to the Online Policy Group v. Diebold case that launched Students for Free Culture (née FreeCulture.org), I have remained on the Core Team and supported newly-forming chapters.
           \item Participate as a peer with the board, recently helping to plan and execute the 2008 Free Culture Conference in Berkeley.
	   \end{itemize}
           
           
	 \item 
	   \begin{tabular*}{6in}{l@{\extracolsep{\fill}}r}
	     \textbf{Chapter chair and lab admin} & May 2005 - May 2007\\
	     JHU Association for Computing Machinery & Baltimore, MD
	   \end{tabular*}
	   \begin{itemize}
	   \item As chair from 2005-2006, organized weekly meetings including personally presenting on reverse-engineering an XML-RPC web radio system, inviting corporate speakers, and hosting student teaching sessions.
           \item Originally as chair, and continuing as lab admin from 2006-2007, undertook significant system administration and data recovery tasks on Mac OS X, GNU/Linux, and SunOS on Sparc, i386, and PowerPC architectures.
           \item Tripled average meeting attendance when compared to previous year through strong advertising and engaging meeting topics.
	   \end{itemize}
           
	 \item 
	   \begin{tabular*}{6in}{l@{\extracolsep{\fill}}r}
	     \textbf{Web team leader} & September 2004 - May 2006 \\
	     J-Stream: Hopkins Streaming Media Network & Baltimore, MD
	   \end{tabular*}
	   \begin{itemize}
	   \item Led web team in creating a new video streaming service for JHU community members; customized Drupal and built creative network architecture to distribute content across servers.
	   \end{itemize}
           
	 \item 
	   \begin{tabular*}{6in}{l@{\extracolsep{\fill}}r}
	     \textbf{Wiki maintainer} & April 2003 - May 2006 \\
	     Xbox-Linux Project
	   \end{tabular*}
	   \begin{itemize}
	   \item Created and hosted a wiki to organize user-created documentation; collaboratively maintain the current MediaWiki engine that powers the Xbox-Linux.org website.
	   \end{itemize}
           

	 \end{itemize}

         
         {\large\textbf{Interests and extracurriculars}}
         \begin{itemize}
           \item 2004-ongoing: Annually participate in MIT Mystery Hunt.
           \item College clubs: Entertainers club (juggling), pep
             band, choral society, ECCO chamber chorale.
           \item Co-founded and maintained ``Hopkins Weblogs'' and
             ``JhuWiki'' campus community web services.
         \end{itemize}
         
	 
	 {\large \textbf{Skills}}
	 \begin{description}
	 \item[Languages:]
	   Fluent: Python, XML (SAX/DOM), English, RDF. \\
	   Working knowledge: C, C++, French, (X)HTML, CSS, \LaTeX{}, Java, PHP, Common Lisp, \begin{tt}bash\end{tt}, Prolog. \\
             Basic skills: Hindi, \begin{tt}perl\end{tt}.
               
	     \item[Operating Systems:]
	       Extensive GNU/Linux experience in Debian, Ubuntu, and Fedora.  Strong background in Mac OS X and Windows NT/XP.
	     \item[Applications:]
	       Extensive experience with OpenSSH, Apache, mod\_rewrite, git, Subversion, MediaWiki.
	 \end{description}
\end{document}
